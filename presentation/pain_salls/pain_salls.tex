\documentclass[dvipdfmx]{beamer}\usepackage[]{graphicx}\usepackage[]{color}
% maxwidth is the original width if it is less than linewidth
% otherwise use linewidth (to make sure the graphics do not exceed the margin)
\makeatletter
\def\maxwidth{ %
  \ifdim\Gin@nat@width>\linewidth
    \linewidth
  \else
    \Gin@nat@width
  \fi
}
\makeatother

\definecolor{fgcolor}{rgb}{0.345, 0.345, 0.345}
\newcommand{\hlnum}[1]{\textcolor[rgb]{0.686,0.059,0.569}{#1}}%
\newcommand{\hlstr}[1]{\textcolor[rgb]{0.192,0.494,0.8}{#1}}%
\newcommand{\hlcom}[1]{\textcolor[rgb]{0.678,0.584,0.686}{\textit{#1}}}%
\newcommand{\hlopt}[1]{\textcolor[rgb]{0,0,0}{#1}}%
\newcommand{\hlstd}[1]{\textcolor[rgb]{0.345,0.345,0.345}{#1}}%
\newcommand{\hlkwa}[1]{\textcolor[rgb]{0.161,0.373,0.58}{\textbf{#1}}}%
\newcommand{\hlkwb}[1]{\textcolor[rgb]{0.69,0.353,0.396}{#1}}%
\newcommand{\hlkwc}[1]{\textcolor[rgb]{0.333,0.667,0.333}{#1}}%
\newcommand{\hlkwd}[1]{\textcolor[rgb]{0.737,0.353,0.396}{\textbf{#1}}}%
\let\hlipl\hlkwb

\usepackage{framed}
\makeatletter
\newenvironment{kframe}{%
 \def\at@end@of@kframe{}%
 \ifinner\ifhmode%
  \def\at@end@of@kframe{\end{minipage}}%
  \begin{minipage}{\columnwidth}%
 \fi\fi%
 \def\FrameCommand##1{\hskip\@totalleftmargin \hskip-\fboxsep
 \colorbox{shadecolor}{##1}\hskip-\fboxsep
     % There is no \\@totalrightmargin, so:
     \hskip-\linewidth \hskip-\@totalleftmargin \hskip\columnwidth}%
 \MakeFramed {\advance\hsize-\width
   \@totalleftmargin\z@ \linewidth\hsize
   \@setminipage}}%
 {\par\unskip\endMakeFramed%
 \at@end@of@kframe}
\makeatother

\definecolor{shadecolor}{rgb}{.97, .97, .97}
\definecolor{messagecolor}{rgb}{0, 0, 0}
\definecolor{warningcolor}{rgb}{1, 0, 1}
\definecolor{errorcolor}{rgb}{1, 0, 0}
\newenvironment{knitrout}{}{} % an empty environment to be redefined in TeX

\usepackage{alltt}
\usetheme{Boadilla}
%\usepackage{helvet}
\usepackage{float}
\usepackage{standalone}
\usepackage{tikz}
\usepackage{booktabs}
\usepackage{dcolumn} % to align numeric values at decimal mark in stargazer
%\usepackage{geometry}
%\geometry{a4paper,portrait,margin=1in} % graph extended to the defined width. graph size need to be defined in global environment
%\AtBeginSection[] % show highlighted table of contents for each section
%{
%  \begin{frame}
%    \frametitle{Table of Contents}
%    \tableofcontents[currentsection]
%  \end{frame}
%}

\title{Neighborhood environment and chronic pain among oldr adults}
\subtitle{Swedish Annual Level of Living Survey}
\author{Kenta Okuyama\inst{1,2}}
\institute{\inst{1}Center for Primary Health Care Research \\
Lund University\and \inst{2}Center for Community-based Healthcare Research and Education \\
Shimane University}
\date{\today}
\logo{
	\includegraphics[height=1.0cm]{/home/kenta/Dropbox/research_projects/research_proposal/logo/lund_RGB.png}~
	%\includegraphics[height=1.0cm]{/home/kenta/Dropbox/Logo/logo_shimaneUniv.png}
}

%%%%%%%%%%%%%%%%%%%%%%%%%%%
\IfFileExists{upquote.sty}{\usepackage{upquote}}{}
\begin{document}
\begin{frame}
	\titlepage
\end{frame}

%%%%%%%%%%%%%%%%%%%%%%%%%%%
%\begin{frame}
%	\frame{Table of Contents}
%	\tableofcontents
%\end{frame}

%%%%%%%%%%%%%%%%%%%%%%%%%%%
\section{Introduction}
\begin{frame}
	\frametitle{Introduction}
	\begin{itemize}
		\item Facts for older adults (ageing population)
			\begin{itemize}
				\item World population is ageing.
			\end{itemize}
	\end{itemize}
\end{frame}

%%%%%%%%%%%%%%%%%%%%%%%%%%%
\begin{frame}
	\frametitle{Introduction (Cont.)}
	\begin{itemize}
		\item Quality of life of older adults (endpoint from pain)
			\begin{itemize}
				\item Maintain quality of life (independent).
				\item Maintain physical function.
				\item Physical disorders, especially injuries from fall. Mental disorders, especially dementia.
				\item Pain, is one of the biggest factor for falls and mental disorders.
			\end{itemize}
	\end{itemize}
\end{frame}

%%%%%%%%%%%%%%%%%%%%%%%%%%%
\begin{frame}
	\frametitle{Introduction (Cont.)}
	\begin{itemize}
		\item Fact of pain.
			\begin{itemize}
				\item Pain is a factor for physical function decline.
				\item Associated healthcare cost with pain.
				\item Current issues with pain, mainly treatment.
				\item 
			\end{itemize}
	\end{itemize}
\end{frame}


%%%%%%%%%%%%%%%%%%%%%%%%%%%
\begin{frame}
	\frametitle{Introduction (Cont.)}
	\begin{itemize}
		\item Neighborhood environment
			\begin{itemize}
				\item Some individual risk factors have been found for pain.
			\end{itemize}
	\end{itemize}
\end{frame}


%%%%%%%%%%%%%%%%%%%%%%%%%%%
\begin{frame}
	\frametitle{Introduction (Cont.)}
	\begin{itemize}
		\item Why neighborhood study
			\begin{itemize}
				\item Bring imapacts on population.
			\end{itemize}
	\end{itemize}
\end{frame}

%%%%%%%%%%%%%%%%%%%%%%%%%%%
\section{Objectives}
\begin{frame}
	\frametitle{Objectives}
	\begin{block}{Objectives}
		\begin{itemize}
			\item To investigate whether neighborhood deprivation is associated with chronic pain among older adults.
		\end{itemize}
	\end{block}
	\begin{description}
		\item[Hypothesis] \mbox{}\par
			\begin{itemize} 
				\item Older adults living in deprived areas would have higher risk of developing chronic pains.
			\end{itemize} 
		\item[Implication] \mbox{}\par
			\begin{itemize} 
				\item The findings leads to further investigations for what modifiable factors lie in neighborhood deprivation and pain.
			\end{itemize} 
	\end{description}
\end{frame}

%%%%%%%%%%%%%%%%%%%%%%%%%%%
\section{Methods}
\begin{frame}
	\frametitle{Methods}
	\begin{description}
		\item[Study areas] \mbox{}\par
			\begin{itemize} 
				\item Entire sweden.
			\end{itemize} 
				\item[Study subjects] \mbox{}\par
			\begin{itemize} 
				\item Those who participated in Swedish Annual Level Living Survey in 1988.
			\end{itemize} 
	\end{description}

\end{frame}

%%%%%%%%%%%%%%%%%%%%%%%%%%%
\begin{frame}
	\frametitle{Methods (Cont.)}
	\resizebox{\textwidth}{!}{
		\begin{tabular}{ c l p{5cm} }
			\hline
			Type & Name & Description\\
			\hline \hline
			Outcome & Anxiety & Self-reported anxiety asked via interview. \\
			Exposure & Urbanization & Neighborhood size, ex. urban, rural. More details if possible. \\
							 & Social interactions & Frequencies of social interactions asked via interview. \\
							 & Internet use & Frequencies of internet use for social interactions asked via interview.\\
			Covariates & Basic characteristics & Age, gender, immigration status. \\
								 & Socio-economic status & Education, occupation, marital status, income.\\
								 & Psychosocial work environment& Job demands, decision making, supports.\\
								 \hline
		\end{tabular}
	}
\end{frame}

\end{document}
