\documentclass[dvipdfmx]{beamer}\usepackage[]{graphicx}\usepackage[]{color}
% maxwidth is the original width if it is less than linewidth
% otherwise use linewidth (to make sure the graphics do not exceed the margin)
\makeatletter
\def\maxwidth{ %
  \ifdim\Gin@nat@width>\linewidth
    \linewidth
  \else
    \Gin@nat@width
  \fi
}
\makeatother

\definecolor{fgcolor}{rgb}{0.345, 0.345, 0.345}
\newcommand{\hlnum}[1]{\textcolor[rgb]{0.686,0.059,0.569}{#1}}%
\newcommand{\hlstr}[1]{\textcolor[rgb]{0.192,0.494,0.8}{#1}}%
\newcommand{\hlcom}[1]{\textcolor[rgb]{0.678,0.584,0.686}{\textit{#1}}}%
\newcommand{\hlopt}[1]{\textcolor[rgb]{0,0,0}{#1}}%
\newcommand{\hlstd}[1]{\textcolor[rgb]{0.345,0.345,0.345}{#1}}%
\newcommand{\hlkwa}[1]{\textcolor[rgb]{0.161,0.373,0.58}{\textbf{#1}}}%
\newcommand{\hlkwb}[1]{\textcolor[rgb]{0.69,0.353,0.396}{#1}}%
\newcommand{\hlkwc}[1]{\textcolor[rgb]{0.333,0.667,0.333}{#1}}%
\newcommand{\hlkwd}[1]{\textcolor[rgb]{0.737,0.353,0.396}{\textbf{#1}}}%
\let\hlipl\hlkwb

\usepackage{framed}
\makeatletter
\newenvironment{kframe}{%
 \def\at@end@of@kframe{}%
 \ifinner\ifhmode%
  \def\at@end@of@kframe{\end{minipage}}%
  \begin{minipage}{\columnwidth}%
 \fi\fi%
 \def\FrameCommand##1{\hskip\@totalleftmargin \hskip-\fboxsep
 \colorbox{shadecolor}{##1}\hskip-\fboxsep
     % There is no \\@totalrightmargin, so:
     \hskip-\linewidth \hskip-\@totalleftmargin \hskip\columnwidth}%
 \MakeFramed {\advance\hsize-\width
   \@totalleftmargin\z@ \linewidth\hsize
   \@setminipage}}%
 {\par\unskip\endMakeFramed%
 \at@end@of@kframe}
\makeatother

\definecolor{shadecolor}{rgb}{.97, .97, .97}
\definecolor{messagecolor}{rgb}{0, 0, 0}
\definecolor{warningcolor}{rgb}{1, 0, 1}
\definecolor{errorcolor}{rgb}{1, 0, 0}
\newenvironment{knitrout}{}{} % an empty environment to be redefined in TeX

\usepackage{alltt}
\usetheme{Boadilla}
%\usepackage{helvet}
\usepackage{float}
\usepackage{standalone}
\usepackage{tikz}
\usepackage{booktabs}
\usepackage{dcolumn} % to align numeric values at decimal mark in stargazer
%\usepackage{geometry}
%\geometry{a4paper,portrait,margin=1in} % graph extended to the defined width. graph size need to be defined in global environment
%\AtBeginSection[] % show highlighted table of contents for each section
%{
%  \begin{frame}
%    \frametitle{Table of Contents}
%    \tableofcontents[currentsection]
%  \end{frame}
%}

\title{Research Proposal}
\subtitle{Longitudinal change of anxiety and individual and neighborhood factors among Swedish population}
\author{Kenta Okuyama\inst{1,2}}
\institute{\inst{1}Center for Primary Health Care Research \\
Lund University\and \inst{2}Center for Community-based Healthcare Research and Education \\
Shimane University}
\date{\today}
\logo{
	\includegraphics[height=1.0cm]{/home/kenta/Dropbox/research_projects/research_proposal/logo/lund_RGB.png}~
	%\includegraphics[height=1.0cm]{/home/kenta/Dropbox/Logo/logo_shimaneUniv.png}
}

%%%%%%%%%%%%%%%%%%%%%%%%%%%
\IfFileExists{upquote.sty}{\usepackage{upquote}}{}
\begin{document}
\begin{frame}
	\titlepage
\end{frame}

%%%%%%%%%%%%%%%%%%%%%%%%%%%
%\begin{frame}
%	\frame{Table of Contents}
%	\tableofcontents
%\end{frame}

%%%%%%%%%%%%%%%%%%%%%%%%%%%
\section{Introduction}
\begin{frame}
	\frametitle{Introduction}
	\begin{itemize}
		\item The prevalence of \textbf{anxiety has been increasing for 25 years} in Sweden (\textit{Calling, 2017}).
		\item \textbf{Especially among young females (16-23 years),} 1/3 experienced anxiety in 2005, and it has increased dramatically.
		\item Anxiety is a predictor of severe psychiatric disorders (\textit{Weitoft, 2005}).
		\item Several factors were found to be associated with the prevalence of anxiety:
			\begin{itemize}
				\item Urbanization
				\item Leisure time physical activity
				\item Smoking
				\item Marital status
				\item Neighborhood deprivation, social network, employnment, immigration status (\textit{Lofors, 2006})
			\end{itemize}
	\end{itemize}
\end{frame}

%%%%%%%%%%%%%%%%%%%%%%%%%%%
\begin{frame}
	\frametitle{Introduction}
	\begin{itemize}
		\item However, "the reasons of increased self-reported anxiety" are unknown
		\item For example, the prevalence of leisure time PA has been increasing for 25 years (\textit{Leijon, 2015}), in parallel with the increase of the prevalence of anxiety.
		\item Potential factors:
			\begin{itemize}
				\item Increased unemployment rate
				\item Incraesed awareness of mental health and decreased stigma
				\item \textbf{Urbanization}
				\item \textbf{Social interactions}
				\item \textbf{Internet use}
			\end{itemize}
	\end{itemize}
	\begin{block}{Objective}
		\begin{itemize}
			\item To investigate \textbf{how direct (physical) and indirect (virtual) social interactions affect on anxiety by different neighborhood settings}.
		\end{itemize}
	\end{block}
\end{frame}

%%%%%%%%%%%%%%%%%%%%%%%%%%%
\section{Hypothesis}
\begin{frame}
	\frametitle{Hypothesis}
%%% left column	
		\begin{itemize}
			\item Direct social interactions affect on self-reported anxiety \textbf{positively, but the effects differ by neighborhood settings}.
			\item Internet-use for social interactions affect on self-reported anxiety \textbf{positively among those in non-urban areas, but negatively among those in urban areas}.
		\end{itemize}

		\begin{block}{Significance}
			\begin{itemize}
				\item \textbf{Regulation or utilization of internet can be considered as effective interventions for mental health in different geographical settings}
			\end{itemize}
		\end{block}

\end{frame}

%%%%%%%%%%%%%%%%%%%%%%%%%%%
\section{Methods}
\begin{frame}
	\frametitle{Methods}
			%\caption{Description of variables to be analyzed}
			%\centering
	\resizebox{\textwidth}{!}{
			\begin{tabular}{ c l p{5cm} }
				\hline
				Type & Name & Description\\
				\hline \hline
				Outcome & Anxiety & Self-reported anxiety asked via interview. \\
				Exposure & Urbanization & Neighborhood size, ex. urban, rural. More details if possible. \\
								 & Social interactions & Frequencies of social interactions asked via interview. \\
								 & Internet use & Frequencies of internet use for social interactions asked via interview.\\
				Covariates & Basic characteristics & Age, gender, immigration status. \\
									 & Socio-economic status & Education, occupation, marital status, income.\\
									 & Psychosocial work environment& Job demands, decision making, supports.\\
									 \hline
			\end{tabular}
		}
\end{frame}

