\documentclass{beamer}
\usetheme{Boadilla} % the theme can be combined with colortheme
%\usefonttheme{structuresmallcapsserif} % change header font - structurebold, structurebolditalic, structuresmallcapsserif, structureitalicsserif, serif, default.
\usepackage{helvet} % change entire font - mathptmx, helvet, bookman, chancery, charter, mathptm, newcent, palatino, pifont, utopia.

\AtBeginSection[] % show highlighted table of contents for each section
{
  \begin{frame}
    \frametitle{Table of Contents}
    \tableofcontents[currentsection]
  \end{frame}
}


\title{World population ageing}
\subtitle{What is a problem?}
\author{Kenta Okuyama\inst{1}}
\institute{\inst{1}Center for Primary Health Care Research \\
Lund University}
\date{\today}
\logo{\includegraphics[height=1.5cm]{../../logo/lund_RGB.png}} % logo will be appeared in each slide


\begin{document}

\begin{frame}
\titlepage
\end{frame}

\section{Introduction} % to make it visible in table of contents
\begin{frame} % \alert will highlight the part depending on the theme. \pause will add animation.
\frametitle{Introduction}
\begin{itemize}
	\item Most people recognize the term \itembf{"population ageing"} 
	\item However, most people might not be able to explain what are problems due to population ageing
	\item This lecture will go through what is currently happening in the world in terms of ageing based on statistics and associated problems and how to deal with them based on the \itembf{World Population Ageing 2015 Report by United Nations}
\end{itemize}
\end{frame}

\section{Aims}
\begin{frame} % <> will do the same as \pause
\frametitle{Aims}
\begin{columns}
	\column{0.5\textwidth}
	\begin{description}
		\item[Aim 1]<1-> Examine the longitudinal effect of neighborhood environment on obesity
		\item[Aim 2]<2-> Examine whether effect of neighborhood environment on obesity is confounded/modified by neighborhood deprivation 
	\end{description}
\end{columns}
\end{frame}

\section{Methods}
\begin{frame}
\frametitle{Methods}
\end{frame}





\end{document}
