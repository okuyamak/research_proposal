\documentclass[12]{article}



\title{Research proposal: Longitudinal effect of neighbordhood physical environment on obesity among Swedish adults}
\date{\today}
\author{Kenta Okuyama}


\begin{document}
\maketitle
\pagenumbering{arabic}
  
\section{Aims} %Biggest section
\paragraph{Aim 1}
Examine the longitudinal association between neighborhood physical environment (fast food outlets and physical activity facilities) and obesity.
\paragraph{Aim 2}
Examine whether the association between neighborhood physical environment and obesity is confounded or modified by individual and neighborhood socio-economic status.

\section{Methods}
\subsection{Study sample}
\paragraph{} Nationwide sample of men and women age between 20 - 55 years old from a national Swedish registers. Information about women's weight, height and BMI will be obtained from the Swedish Medical Birth Resgister, which is a register of all pregnancies, prenatal care and birth records for all mothers and children in Sweden since 1973. Information about men's weight, height and BMI will be obtained from the Military Conscription Register, which includes a structured and standardized medical assessment of all Swedish men since 1969. Baseline will be set for 2005, as a ready to use neighborhood measures (i.e., density of fast food outlets, and physical activity facilities) are available at this time period. Follow-up period will be until 2015, which is the last year of available follow-up data.
\subsection{Outcome}
\paragraph{}    
Incidence of obesity will be identified by a hospital or out-patient diagnosis of obesity during study period. Hospital Discharge Register and Out-Patient Register will be used to collect the diagnosis of obesity, which can be linked by serial number of men and women's cohort datasets.
\subsection{Exposure}
\paragraph{}
Density of fast food outlets (e.g., pizzerias, and hamburger joints) and physical activity facilities (e.g., swimming pools, gyms, ski facilities) calculated by geographic information system (GIS) would be used as primary exposure variables for obesity. Neighborhood space will be defined by administrative boundary (Small Area Market Statistics (SAMS)), and counts of facilities within each boundary will be used as measures of density. Each exposure will be examined by separate models to predict the effect of each exposure on obesity. 

\subsection{Covariates}
\paragraph{}
Potenitial confounding variables and effect modifiers, i.e., age, gender, BMI, immigration status, family income, educational attainment, occupation, and neighborhood deprivation will be controlled (Table 1).  

\begin{table}[h!]
		    \caption{Description of variables to be analyzed}
		    \label{table:aim1}
		      \centering
		      \begin{tabular}{ c l p{5cm} }
			    \hline
			    Type & Name & Description\\
			    \hline \hline
			    Outcome & Obesity & Incidence of obesity identified from the Hospital Discharge Register and Out-Patient Register.\\
			    Exposure & Food environment & Density of fastfood outlets (e.g., pizzerias, and hamburger joints).\\
			             & Physical activity environment &  Density of physical activity facilities (e.g., swimming pools, gyms, ski facilities).\\
			    Covariates & Basic characteristics & Age, gender, BMI, and immigration status. \\
			               & Socio-economic status & Education, occupation, and income.\\
				            & Neighborhood deprivation & Neighborhood deprivation index.\\
          \hline
			    \end{tabular}
			  \end{table}
\subsection{Statistical analysis}
\paragraph{}
Cox-propotional hazard model will be applied for the incidence of obesity by neighborhood exposure variables. 
	    

\end{document}
