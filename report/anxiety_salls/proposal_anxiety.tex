\documentclass{article}
%\usepackage[dvipdfmx]{graphicx} % this is for insert image
%\usepackage[dvipdfmx]{color}
%\usepackage{subcaption} %this is for multiple images
\usepackage{hyperref}
\usepackage{float}
\usepackage{standalone}
\usepackage{tikz}
%\usetikzlibrary{positioning}
\hypersetup{
    colorlinks=true,
    linkcolor=blue,
    filecolor=magenta,      
    urlcolor=cyan,
}


\title{Research Proposal}
%\subtitle{The study to investigate individual and neighborhood factors associated with increase of longitudinal self-reported anxiety among Swedish population}
\date{\today}
\author{Kenta Okuyama}


\begin{document}
%\[\pagenumbering{gobble} %Gobble pagenumber in title page
\maketitle
%\setcounter{tocdepth}{1}
%\tableofcontents
%\newpage
%\pagenumbering{arabic}

\section{Introduction} 
\begin{itemize}
	\item The prevalence of anxiety have been increasing over the past 25 years (Calling, 2017).
	\item Several factors have been found to be associated with this longitudinal change of anxiety.
	\item For example, those who engage in regular exercise have had lower increase of the prevalence of anxiety.
	\item However, the prevalence of regular exercise has been increasing over the past 25 years (Leijon, 2015).
	\item It suggests that although regular exercise is important to reduce the risk of anxiety, its effect may not be strong in the population level.
	\item Exercise and eating are 2 major health related behaviors of humans, but usually difficult to be modified.
	\item Socializing is another behavior which has been found important to our health.
	\item It may also be difficult for a certain groups of people, but there have been findings that liviing in areas with high social capital would be beneficial no matter what they have social interactions.
	\item From these backgrounds, this study aims to examine the associations between neighborhood level social relations and longitudinal change of self-reported anxiety.
\end{itemize}

\section{Aims}

\section{Methods}
		\begin{table}[H]
			\caption{Description of variables to be analyzed}
			\centering
			\begin{tabular}{ c l p{5cm} }
				\hline 
				Type & Name & Description \\
				\hline \hline
				Outcome & Sarcopenia (categorical) & 
				\textbf{Male} - max grip strengh \textless26kg and SMI \textless7kg/m$^2$. \textbf{Female} - max grip strength \textless18kg and SMI \textless5.7kg/m$^2$.\\
					& Max grip strength (continuous) & Max grip strength kg after two times trial. \\
					& SMI (continuous) & SMI kg/m$^2$ measured by bio-impedance analysis. \\
				\hline
				Exposure& Mean land slope & Mean land slope (degree in angular unit) within 1000m network buffer from the residential point. \\
					& Bus stop density & Number of bus stops within 1000m network buffer. \\
					& Residential density & Number of residences within 1000m network buffer. \\
					& Intersection density & Number of intersections within 1000m network buffer. \\
				\hline
				Covariates & Basic characteristics & Age, gender, BMI, smoking, drinking. \\
					   & Socio-demographic variables & Driving license, education. \\
					   & Comobidities & Musclue skeletal diseases, Stroke, Cardiovascular disease. \\
				\hline
			\end{tabular}
		\end{table}

		



\end{document}
